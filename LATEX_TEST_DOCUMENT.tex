\documentclass[12pt,a4paper]{article}
\usepackage[utf8]{inputenc}
\usepackage{amsmath}
\usepackage{amsfonts}
\usepackage{amssymb}
\usepackage{graphicx}
\usepackage{geometry}
\geometry{margin=1in}

\title{LaTeX Editor Test Document}
\author{Test User}
\date{\today}

\begin{document}

\maketitle

\section{Introduction}

This is a test document to verify the LaTeX editor functionality. It includes various LaTeX features to test both the MathJax preview and PDF compilation.

\subsection{Basic Text Formatting}

Here is some \textbf{bold text}, \textit{italic text}, and \underline{underlined text}. We can also use \texttt{monospace text} for code.

\section{Mathematical Expressions}

\subsection{Inline Math}

Einstein's famous equation is $E = mc^2$. The quadratic formula is $x = \frac{-b \pm \sqrt{b^2 - 4ac}}{2a}$.

\subsection{Display Math}

The Pythagorean theorem states:
\[
a^2 + b^2 = c^2
\]

The Euler's identity is one of the most beautiful equations in mathematics:
\[
e^{i\pi} + 1 = 0
\]

The integral of a function can be written as:
\[
\int_{a}^{b} f(x) \, dx = F(b) - F(a)
\]

\subsection{Complex Mathematical Expressions}

The Schrödinger equation:
\[
i\hbar\frac{\partial}{\partial t}\Psi(\mathbf{r},t) = \hat{H}\Psi(\mathbf{r},t)
\]

Maxwell's equations in differential form:
\begin{align}
\nabla \cdot \mathbf{E} &= \frac{\rho}{\varepsilon_0} \\
\nabla \cdot \mathbf{B} &= 0 \\
\nabla \times \mathbf{E} &= -\frac{\partial \mathbf{B}}{\partial t} \\
\nabla \times \mathbf{B} &= \mu_0\mathbf{J} + \mu_0\varepsilon_0\frac{\partial \mathbf{E}}{\partial t}
\end{align}

\section{Lists}

\subsection{Unordered List}

\begin{itemize}
    \item First item
    \item Second item
    \item Third item with \textbf{bold text}
    \item Fourth item with math: $f(x) = x^2 + 3x + 2$
\end{itemize}

\subsection{Ordered List}

\begin{enumerate}
    \item First step
    \item Second step
    \item Third step
    \item Fourth step
\end{enumerate}

\section{Tables}

Here is a simple table:

\begin{table}[h]
\centering
\begin{tabular}{|c|c|c|}
\hline
\textbf{Column 1} & \textbf{Column 2} & \textbf{Column 3} \\
\hline
Row 1, Col 1 & Row 1, Col 2 & Row 1, Col 3 \\
Row 2, Col 1 & Row 2, Col 2 & Row 2, Col 3 \\
Row 3, Col 1 & Row 3, Col 2 & Row 3, Col 3 \\
\hline
\end{tabular}
\caption{A sample table}
\end{table}

\section{Special Characters and Symbols}

Greek letters: $\alpha, \beta, \gamma, \delta, \epsilon, \theta, \lambda, \mu, \pi, \sigma, \phi, \omega$

Mathematical operators: $\sum, \prod, \int, \oint, \partial, \nabla$

Relations: $\leq, \geq, \neq, \approx, \equiv, \sim$

\section{Conclusion}

This document demonstrates various LaTeX features including:
\begin{itemize}
    \item Document structure and sections
    \item Text formatting
    \item Mathematical expressions (inline and display)
    \item Lists (ordered and unordered)
    \item Tables
    \item Special symbols and characters
\end{itemize}

The LaTeX editor should be able to compile this document to PDF successfully.

\end{document}
